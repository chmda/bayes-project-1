 \section{Introduction}
 Le but de ce projet est de classer la performance de certaines écoles Londoniennes \footnote{\url{https://github.com/chmda/bayes-project-1}}.
 
  \subsection{Jeu de données}
    On dispose d'un jeu de données comprenant les notes de 1978 élèves (une note par élève) dans 38 écoles différentes. On considère que la note de l'élève dépend des autres données que nous disposons, qui sont : 
    \begin{itemize}
        \item Le genre de l'élève.
        \item L'école de l'élève.
        \item Une note de lecture de l'élève (LRT : London Reading Test).
        \item Une catégorie de raisonnement de l'élève (allant de 1 à 3) (VR : Verbal Reasoning).
        
            \item Le genre de l'école (Ecole pour fille, pour garçon, ou mixte).
        \item Le type du bâtiment de l'école (Church of England, Roman Catholic, State school ou autres).
    \end{itemize}
    
    
  \subsection{Modèle mathématique}
    On utilise un modèle hierarchique. Notons $Y_{ij}$ la note du l'élève $i$ dans l'école $j$.
    Dans notre modèle. $Y_{ij} \sim \mathcal{N}( \mu_{ij}, \tau_{ij})$. 
    \begin{itemize}
        \item Le logarithme de $\tau_{ij}$ est modélisé comme suivant une fonction linéaire de la note du test LRT. On note $\log (\tau_{ij}) = \theta + \phi LRT_{ij}$
        \item $\mu_{ij}$ est définie comme égale à 
        
        $\alpha_{1j} 
        + \alpha_{2j}\cdot LRT_{ij} 
        + \alpha_{3j}\cdot VR_{1ij} 
        + \beta_{1}\cdot LRT^{2}_{ij} 
        + \beta_{2}\cdot VR_{2ij} 
        + \beta_{3}\cdot Girl_{ij} 
        + \beta_{4}\cdot GirlSchool_{j} 
        + \beta_{5}\cdot BoySchool_{j} 
        + \beta_{6}\cdot CESchool_{j} 
        + \beta_{7}\cdot RCSchool_{j} 
        + \beta_{8}\cdot OtherSchool_{j}$
    \end{itemize}
    
    Les distributions à priori sont les suivantes : 
    \begin{itemize}
        \item Les variables aléatoires $\beta_k$ où $k \in [1,8]$, ainsi que $\theta$ et $\phi$ suivent des lois normales centrées et indépendantes, de précision 0.0001.
        \item Le vecteur aléatoire $(\alpha_{kj})_k$ pour $k \in [1,3]$, suit une loi normal multivariée $\alpha_{\cdot j} \sim \mathcal{N}( \gamma, \Sigma)$. Où $\gamma$ suit une loi normal multivariée non informative, et $T = \Sigma^{-1}$ suit une distribution de Wishart.
    \end{itemize}
 
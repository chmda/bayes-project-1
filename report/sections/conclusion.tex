\section{Conclusion}
\subsection{Résultats}
Pour obtenir une estimation des paramètres, on exécute la commande suivante :  \\
python main.py data\textbackslash data.json data\textbackslash init1.json \\
On obtient les valeurs suivantes pour les paramètres : \\
    \begin{table}[]
    \begin{tabular}{|c|c|c|c|c|c|}
    \hline
    coefficient       & mean      & std      & q2.5pc    & median    & q97.5pc  \\ \hline
    beta{[}1{]}  & 1.08e-03  & 6.87e-04 & -5.86e-04 & 1.08e-3   & 2.03e-3  \\ \hline
    beta{[}2{]}  & 3.69e-01  & 1.34e-01 & 1.22e-01  & 3.69e-01  & 6.43e-01 \\ \hline
    beta{[}3{]}  & 2.73e-01  & 8.28e-02 & 1.03e-01  & 2.73e-01  & 4.29e-01 \\ \hline
    beta{[}4{]}  & 7.09e+00  & 3.05e+00 & 2.28e+00  & 7.09e+00  & 1.51e+01 \\ \hline
    beta{[}5{]}  & 3.64e-01  & 3.81e-01 & -2.75e-01 & 3.64e-01  & 1.36e+00 \\ \hline
    beta{[}6{]}  & -1.96e-01 & 6.68e-01 & -2.21e+00 & -1.96e-01 & 8.42e-01 \\ \hline
    beta{[}7{]}  & -1.09e+00 & 1.56e+00 & -5.15e+00 & -1.09e+00 & 4.92e-01 \\ \hline
    beta{[}8{]}  & -5.51e-01 & 3.35e-01 & -1.09e+00 & -5.51e-01 & 1.66e-01 \\ \hline
    gamma{[}1{]} & 6.74e-03  & 1.01e+00 & -1.99e+00 & 6.74e-03  & 1.99e+00 \\ \hline
    gamma{[}2{]} & 1.93e-03  & 9.92e-01 & -1.94e+00 & 1.93e-03  & 1.96e+00 \\ \hline
    gamma{[}3{]} & -3.05e-02 & 9.91e-01 & -1.93e+00 & -3.05e-2  & 1.88e+00 \\ \hline
    phi          & -8.19e-03 & 7.02e-03 & -1.57e-02 & -8.19e-03 & 9.37e-03 \\ \hline
    theta        & -3.85e-02 & 2.49e-01 & -8.28e-01 & -3.85e-02 & 2.21e-01 \\ \hline
    \end{tabular}
\end{table} \\

Comparons ces résultats au valeurs annoncées dans l'énoncé :
\begin{enumerate}
    \item $\beta_{2}$,  $\beta_{3}$,  $\beta_{5}$, $\beta_{6}$, $\beta_{8}$ et $\phi$ sont du même ordre de grandeur que leur valeurs données respectives.
    \item $\beta_{1}$, $\gamma_{2}$ ne sont pas de même ordre de grandeur, mais la différence est faible (une puissance de 10).
    \item $\beta_{4}$, $\beta_{7}$, $\gamma_{1}$, $\gamma_{3}$ et $\theta$ ont des valeurs bien différentes (de signe opposés et/ou d'ordre de grandeur différents).
\end{enumerate}
\\
Notons cependant que les valeurs d'écarts-type obtenus sont de même ordre de grandeur, voire plus faibles que les écarts-types données.
\subsection{Conclusion}

En conclusion, les résultats obtenus nous donne une estimation des paramètres. Il faut toutefois avoir conscience des variations par rapport aux résultats attendus.